\documentclass[a4paper,10pt]{article}

% --- Layout (slightly smaller left/right margins) ---
\usepackage[left=0.75in,right=0.75in,top=0.9in,bottom=0.9in]{geometry}

% --- Lists / spacing control ---
\usepackage{enumitem}
\setlength{\parindent}{0pt}
\setlength{\parskip}{0pt}
\renewcommand{\baselinestretch}{0.97}

% Major-item spacing: add space between top-level items, but keep bullets tight
\newcommand{\majoritemsep}{\vspace{6pt}}

% heading
\newcommand{\resheading}[1]{
  \vspace{4pt}
  \noindent\textbf{\large #1}\par
  \vspace{2pt}
  \hrule
  \vspace{4pt}
}

% A simple left-right row (content left, content right)
\newcommand{\lrrow}[2]{%
  \noindent\begin{tabular*}{\textwidth}{@{}l@{\extracolsep{\fill}}r@{}}%
  #1 & #2\\%
  \end{tabular*}%
}

\begin{document}

% =======================
% Contact Information
% =======================
\begin{center}
    {\LARGE \textbf{Qingkai Dong}} \\
    \vspace{2pt}
    \begin{tabular}{llll}
      Email: qingkai.dong@uconn.edu &
      Website: dqksnow.github.io &
      Phone: +1 (860) 931-9075 &
      Address: Storrs, CT, USA
    \end{tabular}
\end{center}

% =======================
% Education (layout matched to screenshot)
% =======================
\resheading{Education}
% PhD
\lrrow{\textbf{University of Connecticut}}{Aug 2023 -- Present}
\lrrow{PhD in Statistics}{Storrs, CT, USA}
\lrrow{Advisors: Prof. HaiYing Wang and Prof. Jun Yan}{Passed the Ph.D. Qualifying Exam}
\majoritemsep

% MS
\lrrow{\textbf{Zhongnan University of Economics and Law}}{Sep 2020 -- May 2023}
\lrrow{MS in Mathematical Statistics}{Wuhan, China}
\noindent Thesis: Model Averaging and Variable Selection for Accelerated Failure Time Models
\majoritemsep

% BS
\lrrow{\textbf{Qingdao University}}{Sep 2016 -- May 2020}
\lrrow{BS in Applied Statistics}{Qingdao, China}

% =======================
% Research Experience
% =======================
\resheading{Research Experience}

% Use itemsep=0pt to avoid extra spacing between bullets
\begin{itemize}[left=0pt,label={},itemsep=0pt,topsep=0pt,parsep=0pt,partopsep=0pt]

  \item Research Assistant (Academia--Industry Collaboration) \hfill Mar 2024 -- Dec 2025 \\
  Servier Pharmaceuticals \& University of Connecticut
  \begin{itemize}[left=12pt,label=\textbullet,itemsep=0pt,topsep=2pt,parsep=0pt,partopsep=0pt]
    \item \textbf{Primary output:} Coauthored a manuscript introducing a predictive-modeling--assisted interim analysis framework for censored time-to-event endpoints; manuscript in preparation for journal submission.
    \item Developed a covariate-informed approach that augments interim data by predicting event times for censored participants, improving estimation of treatment effects and conditional power.
    \item Proposed evaluation metrics for conditional power forecast accuracy and futility decision quality; conducted extensive simulations to characterize when performance improves or degrades under varied censoring, covariate informativeness/type, prediction range, and sample size.
  \end{itemize}
  \majoritemsep

  \item Research Assistant (Statistical Methodology) \hfill Aug 2023 -- Present \\
  University of Connecticut
  \begin{itemize}[left=12pt,label=\textbullet,itemsep=0pt,topsep=2pt,parsep=0pt,partopsep=0pt]
    \item \textbf{Primary output:} Coauthored a methodology paper on rare-feature-aware subsampling for regression models; manuscript in preparation for journal submission.
    \item Developed theory showing that estimation efficiency for rare binary covariate coefficients is driven by the limited number of rare observations, explaining numerical instability and slow convergence.
    \item Introduced a balanced subsampling method that quantifies rarity to ensure adequate representation of rare features without pilot sampling, improving robustness and efficiency.
    \item Implemented the methods in the subsampling R package; validated performance via theoretical results, simulation studies, and real-data applications; produced documentation and reproducible examples.
  \end{itemize}
  \majoritemsep

  \item Research Assistant (Real-World Health Data) \hfill Jul 2024 -- Aug 2024 \\
  University of Connecticut
  \begin{itemize}[left=12pt,label=\textbullet,itemsep=0pt,topsep=2pt,parsep=0pt,partopsep=0pt]
    \item Conducted data preprocessing and statistical analyses on All of Us data to study associations among social determinants of health, frailty index, and accelerated aging in breast cancer patients.
    \item Delivered analysis-ready datasets and reproducible scripts; performed exploratory and confirmatory analyses under missingness and data-quality constraints.
  \end{itemize}
  \majoritemsep

  \item Research Project (Chemical Sensor Data Analytics) \hfill Mar 2024 -- May 2024 \\
  University of Connecticut
  \begin{itemize}[left=12pt,label=\textbullet,itemsep=0pt,topsep=2pt,parsep=0pt,partopsep=0pt]
    \item Analyzed fluorescence response data from porphyrinoid sensors; applied clustering and statistical analysis to improve compound differentiation and sensor selection.
    \item Produced interpretable summaries and visualizations to support experimental decision-making.
  \end{itemize}
  \majoritemsep

  \item Cross-Disciplinary Collaboration (LLM Explainability) \hfill 2025 -- Present \\
  University of Connecticut \& collaborators
  \begin{itemize}[left=12pt,label=\textbullet,itemsep=0pt,topsep=2pt,parsep=0pt,partopsep=0pt]
    \item Coauthored a survey manuscript on time series explainability with an emphasis on LLM-enabled semantic explanations; curated benchmarks and an accompanying repository.
  \end{itemize}

\end{itemize}

% =======================
% Software
% =======================
\resheading{Software \& Open-Source}
\begin{itemize}[left=0pt,label={},itemsep=0pt,topsep=0pt,parsep=0pt,partopsep=0pt]
  \item R package subsampling (CRAN): A statistical computing toolkit for optimal subsampling in big-data modeling. Supports GLMs, quantile regression, and rare event/rare feature regimes.
  \begin{itemize}[left=12pt,label=\textbullet,itemsep=0pt,topsep=2pt,parsep=0pt,partopsep=0pt]
    \item Implemented optimized sampling probability computation targeting variance reduction and prediction accuracy, including practical routines for pilot sampling and two-step designs.
    \item Engineering focus: modular API, documentation/vignettes, reproducible examples, and scalable computation patterns.
  \end{itemize}
\end{itemize}

% =======================
% Teaching Experience
% =======================
\resheading{Teaching Experience}
\begin{itemize}[left=0pt,label={},itemsep=0pt,topsep=0pt,parsep=0pt,partopsep=0pt]
  \item \textbf{Instructor} \hfill Aug 2025 -- May 2026 \\
  University of Connecticut
  \begin{itemize}[left=12pt,label=\textbullet,itemsep=0pt,topsep=2pt,parsep=0pt,partopsep=0pt]
    \item \textbf{STAT 2255:} Statistical Programming --- Introduction to statistical programming via Python, including data types, control flow, object-oriented programming, GUI-driven applications, data wrangling, visualization, and prediction/classification models.
  \end{itemize}
\end{itemize}

% =======================
% Academic Presentations
% =======================
\resheading{Academic Presentations}
\begin{itemize}[left=0pt,label={},itemsep=0pt,topsep=2pt,parsep=0pt,partopsep=0pt]
   \item Poster Presentation - New England Rare Disease Statistics (NERDS) Workshop, Boston, MA \hfill Oct 2025
   \item Poster Presentation - Dahshu Data Science Symposium, Storrs, CT \hfill Oct 2025
\end{itemize}

% =======================
% Academic Service
% =======================
\resheading{Academic Service}
\begin{itemize}[left=0pt,label={},itemsep=0pt,topsep=2pt,parsep=0pt,partopsep=0pt]
   \item \textbf{Reviewer:} \textit{Statistical Papers}; \textit{Sankhya B}; \textit{Journal of Systems Science and Mathematical Sciences (Chinese)}
\end{itemize}

% =======================
% Awards
% =======================
\resheading{Awards}
\begin{itemize}[left=0pt,label={},itemsep=0pt,topsep=2pt,parsep=0pt,partopsep=0pt]
   \item Predoc Fellowship at University of Connecticut \hfill 2025
   \item First-class Scholarship at Zhongnan University of Economics and Law \hfill 2020, 2022
\end{itemize}

% =======================
% Technical Skills (second last)
% =======================
\resheading{Technical Skills}
\begin{itemize}[left=0pt,label={},itemsep=0pt,topsep=2pt,parsep=0pt,partopsep=0pt]
  \item \textbf{Statistics/Methods:} survival analysis; interim analysis \& conditional power; design simulation; big-data subsampling; model averaging; variable selection
  \item \textbf{Programming:} R (package development), Python (machine learning / deep learning), Git, LaTeX
\end{itemize}

% =======================
% Selected Publications (last)
% =======================
\resheading{Selected Publications}
\begin{itemize}[left=0pt,label={},itemsep=0pt,topsep=2pt,parsep=0pt,partopsep=0pt]
  \item Dong Q, Liu B, Zhao H. Weighted Least Squares Model Averaging for Accelerated Failure Time Models. \emph{Computational Statistics and Data Analysis}, 2023.
  \item Zhao H, Dong Q. Variable Selection for Additive Hazards Model with Current Status Data. \emph{Journal of Systems Science and Mathematical Sciences}, 2022.
  \item Zhao H, Liu B, Dong Q. Jackknife Model Averaging of AFT Model with Current Status Data. \emph{Acta Mathematicae Applicatae Sinica}, 2023.
  \item Chen Z, Lucchesi G, Dong Q, Zheng X, Song D, Wen Q, Cheng W, Ni J, Luo D. From Signals to Semantics: A Survey on Time Series Explainability through a Human-Cognitive Lens. \emph{Under review}, 2025.
\end{itemize}

\end{document}
