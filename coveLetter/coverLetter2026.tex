%%% Local Variables:
%%% mode: latex
%%% TeX-master: t
%%% End:

\documentclass[a4paper,10pt]{letter}
\usepackage[utf8]{inputenc}
\usepackage{geometry}
\geometry{margin=1in}

\begin{document}

\begin{letter}{
    Predoctoral Fellowship Committee \\
    Department of Statistics \\
    University of Connecticut}

\opening{Dear Predoctoral Fellowship Committee,}


I am writing to apply for the Predoctoral Fellowship for the Spring
2026 semester. I am a third-year Ph.D. student in the Department of
Statistics who passed the qualifying exam on the first attempt. My
advisors are Professor Haiying Wang and Professor Jun Yan. My current
support for the Fall 2025 semester consists of two 10-hour Research
Assistantships supervised by Professors Wang and Yan.


Over the past year, I have continued to make significant progress in
both methodological development and collaborative research, moving
steadily toward my dissertation and contributing to the broader
statistics community. Specifically, in the following aspects:

\begin{itemize}
\item I publicly released the R package \texttt{subsampling}
on CRAN, which provides optimal subsampling methods for generalized
linear models, quantile regression, and rare binary response
modeling. The package has been downloaded more than 6,000 times
worldwide, and I frequently receive feedback and suggestions from users, reflecting its growing visibility and
impact in the statistical community.
\item I made major progress on the core focus of my dissertation, which addresses the rare feature problem in statistical modeling through a balanced subsampling framework. My research investigates the asymptotic behavior in the presence of rare features and proposes new subampling method to ensure
sufficient representation of rare features. I have made both theoretical and experimental progress on this project.
\item In collaboration with Servier Pharmaceuticals, I have developed machine learning models to improve conditional power estimation for time-to-event endpoints in adaptive clinical trial design. This work led to two poster presentations at the \textit{New England Rare Disease Statistics (NERDS) Workshop} and the \textit{Dahshu Data Science Symposium} in 2025, and will culminate in a paper submission. The current project funding will conclude at the end of this semester.
\item I have also participated in smaller collaborative projects with
other departments at UConn, including a study on the impact of social
determinants of health (SDOH) and frailty index on accelerated aging
in breast cancer patients, and another analyzing fluorescence
responses of porphyrinoid sensors for detecting nitroaromatic
compounds. I also actively contribute to the academic community as a
reviewer for journals such as \textit{Statistical Papers} and
\textit{Sankhya B}.
\end{itemize}


I was fortunate to receive the Predoctoral Fellowship in Spring 2025,
which provided me with valuable research freedom and directly enabled
many of the accomplishments listed above. Looking ahead,
I remain dedicated to pursuing a research-oriented academic
career. The fellowship will allow me to focus more intensively on
completing my dissertation and preparing manuscripts for
publication. I am deeply grateful for the department’s continued
support and for the opportunities that this fellowship provides to
emerging researchers like me.


\closing{Sincerely,}

\vspace{1em}
\noindent\textbf{Qingkai Dong} \\
Ph.D. student \\
Department of Statistics \\
University of Connecticut \\
\texttt{qingkai.dong@uconn.edu}

\end{letter}

\end{document}
